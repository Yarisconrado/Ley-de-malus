\documentclass{article}
\usepackage[utf8]{inputenc}
\usepackage{url}
\usepackage{hyperref}
\usepackage{amsmath}
\usepackage{amsfonts}
\usepackage{amssymb}
\usepackage{graphicx}
\usepackage{float}
\usepackage{lipsum}
\usepackage{multicol}
\setlength{\columnsep}{1cm}%espacio entre columnas
\usepackage{xcolor}
\usepackage{braket}
\usepackage{cancel}
\usepackage{siunitx}
\usepackage{tabularx}
\usepackage[main=english,spanish]{babel}
\usepackage{cite}
\usepackage{fancyhdr} 
\usepackage[left=2cm,right=2cm,top=2cm,bottom=2cm]{geometry}
\usepackage[spanish]{babel}
\usepackage[utf8]{inputenc}
\bibliography{referencias}
\newcommand{\celda}[1]{
  	\begin{minipage}{2.5cm}
		\vspace{5mm}
    		#1
    		\vspace{5mm}
    	\end{minipage}}
        %\definecolor{negro}{rgb}{0.24, 0.58, 0.26}
    

\title{Medición de la velocidad de la luz}
    

\begin{document}
	
	\begin{figure}[H]
		\raggedright
		\includegraphics[scale=0.040]{logouni.png}
	\end{figure}

	
	\begin{center}
		{\Large \textbf{Medida de la velocidad de la luz en un bloque de acrilico}}\\
		\vspace{1.3 mm}
	    \large\textit{Estudiantes: Conrado Yaris, Cuevas Luis, Márquez Pablo, Redondo Yoser.\\ Profesor: Francisco Racedo.}\\ 
		7/03/2025
	\end{center}
\begin{center}
		\textcolor{black}{\rule{150mm}{0.5mm}} % linea
	\end{center}
    \vspace{1.3 mm}


\begin{abstract}
	In this practice, we will focus on studying and understanding the polarization of light and the experimental verification of Malus's law. For this, we will use a polarizer to linearly polarize natural light from a known source, and by using the analyzer (a rotating polarizer), we will analyze how the intensity varies as a function of the angle $ \theta $ between the analyzer and the polarizer. This is to verify whether the equation $ I(\theta) = I_0 \cos^2 (\theta) $ from Malus's law holds.\\  
	Additionally, we will study what happens when the distance between the polarizer and the analyzer is varied while maintaining a fixed angle, as well as polarization by reflection by analyzing and determining Brewster's angle: the angle of incidence at which the reflected light is completely polarized and forms a 90° angle with the refracted ray.\\
	

\end{abstract}
\selectlanguage{spanish}
\begin{abstract}
	En esta practica nos enfocaremos en el estudio y la comprensión de la polarización de la luz y la comprobación experimental de la ley de Malus. Para esto, usaremos un polarizador para polarizar linealmente una luz natural dada por una fuente conocida y mediante el analizador (un polarizador que rota) analizaremos como varia la intensidad en función del ángulo $ \theta $ entre el analizador y el polarizador. Esto para ver si se cumple la ecuación $ I(\theta) = I_0 cos^2 (\theta) $ de la ley de Malus.\\
	Adicionalmente, estudiaremos que ocurre cuando se varia la distancia entre el polarizador y el analizador manteniendo un ángulo fijo y la polarización por reflexión analizando y buscando el ángulo de Brewster: el ángulo de incidencia para el cual la luz reflejada es completamente polarizada y forma un ángulo de 90° con el rayo refractado. \\

   \underline{\textbf{Palabras claves:}}
   Velocidad de la luz, índice de refracción, acrílico, osciloscopio, pulsos de luz, propagación, medio sólido, laboratorio virtual.
\end{abstract}
\vspace{1.5 mm}
\begin{multicols}{2}

	\section*{Introducción}
	\section*{Marco teórico}
	\section*{Desarrollo experimental}
	\subsection*{Practica virtual}
	Para poder realizar el laboratorio virtual de ley de Malus, se siguieron los siguientes pasos:
	\begin{enumerate}
	\item Abrir su navegador de preferencia.
	\item En su  buscador colocar el siguiente link:\url{https://simulabsphysics.com/}
	\item Luego de dar clic, dirigirse a la parte izquierda de la pagina y elegir la opción de óptica.
	\item Al entrar a esa selección encontrara diferentes simulaciones, en este caso a abriremos "Ley de Malus".
	\item Al entrar observaremos, una lampara, un analizador, un polarizador, un sensor y un multímetro. Del lado derecho de la pantalla en el rectángulo azul encontraremos un apartado para variar el angulo del analizador con respecto al eje de transmisión del polarizador y también encontremos un apartado para aumentar o disminuir la intensidad de la lampara.
	\begin{figure}[H]
	\centering
	\includegraphics[scale= 0.3]{virtual.png}
	\caption{Laboratorio virtual de la ley de Malus. Tomado de:\url{https://simulabsphysics.com/optics/vwukcggz}}
	\label{fig:enter-label}
	\end{figure}
	\item Observamos que tenemos una fuente de luz no polarizada, en este caso una lampara, cuando la luz pasa a través del polarizador, observamos como las componentes de la luz paralelas al eje de transmisión de este son las únicas que pasan a través de el,luego de esto, esta luz polarizada vuelve a pasar por el analizador(también es un polarizador),que depende del angulo que forme con respecto al polarizador, veremos como cambia la intensidad de la luz con respecto la variación del angulo del analizador. Luego de realizar lo anterior, la luz pasara a un sensor, que a traves d una resistencia interior y la ley de Ohm, nos permitirá obtener un voltaje en el multímetro.
	\item A través de la variación del angulo del analizador y la intensidad de la luz emitida por la lampara, veremos como afectara en el voltaje registrado y tabularemos los datos.
	\end{enumerate}
	\textit{\textbf{Nota:}}En la parte inferior izquierda, observaremos como el cambio del angulo del analizador impide o permite el paso de la luz.
	\subsection*{Practica presencial}
	Para la realización de la practica experimental se necesitara los siguientes materiales:
	\begin{itemize}
	\item Lampara de luz halógena.
	\item Dos polarizadores lineales.
	\item Un banco óptico.
	\item Un detector
	\item Un lente convergente.
	\item Un multímetro.
	\item Porta lamina.
	\item Lamina.
	\end{itemize}
	\begin{figure}[H]
	\centering
	\includegraphics[scale=0.2]{presencail.jpg}
	\caption{Montaje experimental del laboratorio presencial}
	\label{fig:enter-label}
	\end{figure}
	Con los materiales anteriores realizaremos lo siguiente:
	\begin{enumerate}
	\item Realizar el montaje experimental visto en la fig.(COLOCAR NUMERO DE LA FIGURA).
	\item Antes de empezar el laboratorio se ubica el analizador de tal manera que queden paralelo los ejes de transmisión de ambos polarizadores.
	\item Luego de realizar el paso anterior, variaremos los ángulos entre el analizador y el polarizador y anotaremos los cambios en el voltaje.\
	Este paso lo realizaremos modificando los ángulos en 5° y registraremos los datos.
	\end{enumerate}
	\section*{Resultados y análisis}
	\subsection*{Practica virtual}
	Teniendo en cuenta del desarrollo virtual, podemos obtener los siguientes datos:
	\begin{itemize}
		\item Para una intensidad $I_1$ tenemos los siguientes datos:
		\begin{table}[h]
		\centering
		\begin{tabular}{cc}
			\toprule
			\textbf{Ángulo (°)} & \textbf{Voltaje (V)} \\
			\midrule
			0   & 5.00 \\
			10  & 4.85 \\
			20  & 4.42 \\
			30  & 3.75 \\
			40  & 2.93 \\
			50  & 2.07 \\
			60  & 1.25 \\
			70  & 0.58 \\
			80  & 0.15 \\
			90  & 0.00 \\
			100 & 0.15 \\
			110 & 0.58 \\
			120 & 1.25 \\
			130 & 2.07 \\
			140 & 2.93 \\
			150 & 3.75 \\
			160 & 4.42 \\
			170 & 4.85 \\
			180 & 5.00 \\
			190 & 4.85 \\
			200 & 4.42 \\
			210 & 3.75 \\
			220 & 2.93 \\
			230 & 2.07 \\
			240 & 1.25 \\
			250 & 0.58 \\
			260 & 0.15 \\
			270 & 0.00 \\
			280 & 0.15 \\
			290 & 0.58 \\
			300 & 1.25 \\
			310 & 2.07 \\
			320 & 2.93 \\
			330 & 3.75 \\
			340 & 4.42 \\
			350 & 4.85 \\
			360 & 5.00 \\
			\bottomrule
		\end{tabular}
	\end{table}
	\begin{center}
		Valores de voltaje en función del ángulo para una intensidad de luz no polarizada $I_1$.
	\end{center}
	\item Ahora veremos como cambia el voltaje cuando aumentamos la intensidad de la luz emitida por la lampara:
	\begin{table}[h]
		\centering
		\begin{tabular}{cc}
			\toprule
			\textbf{Ángulo (°)} & \textbf{Voltaje (V)} \\
			\midrule
			0   & 9.00 \\
			10  & 8.73 \\
			20  & 7.95 \\
			30  & 6.75 \\
			40  & 5.28 \\
			50  & 3.72 \\
			60  & 2.25 \\
			70  & 1.05 \\
			80  & 0.27 \\
			90  & 0.00 \\
			100 & 0.27 \\
			110 & 1.05 \\
			120 & 2.25 \\
			130 & 3.72 \\
			140 & 5.28 \\
			150 & 6.75 \\
			160 & 7.95 \\
			170 & 8.73 \\
			180 & 9.00 \\
			190 & 8.73 \\
			200 & 7.95 \\
			210 & 6.75 \\
			220 & 5.28 \\
			230 & 3.72 \\
			240 & 2.25 \\
			250 & 1.05 \\
			260 & 0.27 \\
			270 & 0.00 \\
			280 & 0.27 \\
			290 & 1.05 \\
			300 & 2.25 \\
			310 & 3.72 \\
			320 & 5.28 \\
			330 & 6.75 \\
			340 & 7.95 \\
			350 & 8.73 \\
			360 & 9.00 \\
			\bottomrule
		\end{tabular}
	\end{table}
	Valores de voltaje en función del ángulo con 
	
	\end{itemize}
	\subsection*{Practica presencial}
	En el desarrollo presencial obtenemos los siguientes datos:
	\begin{table}[h]
		\centering
		\begin{tabular}{|c|c|}
			\hline
			
	Analizador$(\theta)$& Voltaje (mV) \\
			\hline
			0°   & 275.5 \\
			5°   & 274.3 \\
			10°  & 273.2 \\
			15°  & 272.7 \\
			20°  & 271.7 \\
			25°  & 269.3 \\
			30°  & 265.9 \\
			35°  & 262.9 \\
			40°  & 259.8 \\
			45°  & 256.1 \\
			50°  & 250.9 \\
			55°  & 245.1 \\
			60°  & 239.7 \\
			65°  & 233.3 \\
			70°  & 225.8 \\
			75°  & 213.4 \\
			80°  & 206.9 \\
			85°  & 200.7 \\
			90°  & 190.5 \\
			\hline
		\end{tabular}
	\end{table}
	
	\begin{center}
	Tabla NUMERO DE TABLA: Datos de voltaje en función del ángulo del analizador (abriendo hacia la derecha)
	\end{center}
	
	\begin{table}[h]
		\centering
		\begin{tabular}{|c|c|}
			\hline
			Ángulo ($\theta$)& Voltaje (mV) \\
			\hline
			0  & 275.5 \\
			5  & 272.6 \\
			10 & 271.9 \\
			15 & 270.5 \\
			20 & 268.7 \\
			25 & 266.4 \\
			30 & 264.1 \\
			35 & 260.9 \\
			40 & 256.9 \\
			45  & 253.4 \\
			50  & 249.7 \\
			55  & 243.1 \\
			60  & 237.0 \\
			65  & 228.9 \\
			70  & 222.1 \\
			75  & 211.9 \\
			80  & 202.6 \\
			85  & 199.9 \\
			90  & 195.6 \\
			\hline
		\end{tabular}
	\end{table}
	
	\begin{center}
	Tabla NUMERO DE TABLA: Datos de voltaje en función del ángulo del analizador(abriendo hacia la izquierda)
	\end{center}
	\end{multicols}

\end{document}